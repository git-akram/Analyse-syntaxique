\documentclass{beamer} 


%======================================================================
% Packages / Options
%======================================================================

\usepackage[utf8]{inputenc}

\usepackage[T1]{fontenc}
\usepackage{listings}
\lstset{language=caml,basicstyle=\ttfamily}
\usepackage{alltt}

% Beamer-specific settings
\mode<presentation>
{
  \usetheme{Darmstadt}
  \useoutertheme{infolines}
}

\AtBeginSection[]
{
  \begin{frame}<beamer>
    \frametitle{Plan}
    \tableofcontents[sectionstyle=show/shaded,subsectionstyle=hide]
  \end{frame}
}

\AtBeginSubsection[]
{
  \begin{frame}<beamer>
    \frametitle{Plan}
    \tableofcontents[sectionstyle=show/hide,subsectionstyle=show/shaded/hide]
  \end{frame}
}

\setbeamertemplate{footline}
{%
\insertpagenumber
\insertshorttitle[width={3cm},center]
\insertshortinstitute[width={3cm},center]
\insertshortdate[width={3cm},center]
}


%======================================================================
% Titlepage
%======================================================================

\title[Progr. Fonction.]{L1 PCP Programmation Fonctionnelle}

\author{Martin Strecker}
\institute{Université de Toulouse/IRIT}
\date{Année 2012/2013} 

\begin{document}

\begin{frame}
  \titlepage
\end{frame}

%======================================================================
\section{Introduction}\label{sec:intro}

%-------------------------------------------------------------
\subsection{Survol: votre premier programme}
%-------------------------------------------------------------

%-------------------------------------------------------------
\begin{frame}[fragile]\frametitle{Premier transparent}

Les transparents ont été faits avec \LaTeX Beamer:

\href{https://bitbucket.org/rivanvx/beamer/wiki/Home}{https://bitbucket.org/rivanvx/beamer/wiki/Home}

\end{frame}

%-------------------------------------------------------------
\subsection{Une autre section}
%-------------------------------------------------------------

%-------------------------------------------------------------
\begin{frame}[fragile]\frametitle{Deuxième transparent}

Les autres commandes \LaTeX peuvent être utilisées comme d'habitude. 

Détails sur 
\href{http://en.wikibooks.org/wiki/LaTeX}{http://en.wikibooks.org/wiki/LaTeX}

\end{frame}



%======================================================================
\section{Fondamenteux}


%-------------------------------------------------------------
\begin{frame}[fragile]\frametitle{}

		\includegraphics[width=0.60\textwidth]{logo-ups.jpg}
\end{frame}


%======================================================================
\section{Comparaison entre les outils}


%-------------------------------------------------------------
\begin{frame}[fragile]\frametitle{}

\begin{alltt}
expr ::= expr '+' expr
      | expr '-' expr
      | '(' expr ')'
      | NUM ;
\end{alltt}

\end{frame}


%======================================================================
\section{Xtext}


%-------------------------------------------------------------
\begin{frame}[fragile]\frametitle{}

\end{frame}


%======================================================================
\section{Conclusion}


%-------------------------------------------------------------
\begin{frame}[fragile]\frametitle{}

\end{frame}

\end{document}

%%% Local Variables: 
%%% mode: latex
%%% TeX-master: t
%%% coding: utf-8
%%% End: 
